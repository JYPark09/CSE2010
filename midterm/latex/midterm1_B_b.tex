\documentclass[a4paper, 11pt]{article}

% packages
\usepackage{kotex}
\usepackage{amssymb}
\usepackage{booktabs, caption}
\usepackage[flushleft]{threeparttable}
\usepackage{fullpage}
\usepackage{algorithm, algpseudocode, algorithmicx}

\title{Report of Question1-B-b}
\author{2019xxxxxx 박준영}
\date{}

\begin{document}
\maketitle


\section{Node Type}
기본적인 Binary Search Tree의 node 구조체와 동일하다.

\section{Insertion}
duplicated key가 insert될 때 left subtree에 작은 값이 insert될 때와 같은 방식으로 insert되도록 하였다.
\\

\begin{algorithm}
	\caption{BST with duplicated left child insertion}
	\begin{algorithmic}
		\Function{Insert}{root, value}
			\If{root is null}
				\State root becomes a new node with value
			\ElsIf{root.value $\geq$ value}
				\State root.leftChild = Insert(root.leftChild, value)
			\ElsIf{root.value $<$ value}
				\State root.rightChild = Insert(root.rightChild, value)
			\EndIf
			
			\State \Return{root}
		\EndFunction
	\end{algorithmic}
\end{algorithm}

\section{Deletion}
기존 Binary Search Tree의 delete와 동일하다.

\section{Finding}
기존 Binary Search Tree의 find와 동일하다.

\end{document}
