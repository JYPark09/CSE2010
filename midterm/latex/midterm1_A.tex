\documentclass[a4paper, 11pt]{article}

% packages
\usepackage{kotex}
\usepackage{amssymb}
\usepackage{booktabs, caption}
\usepackage[flushleft]{threeparttable}
\usepackage{fullpage}
\usepackage{algorithm, algpseudocode, algorithmicx}

\title{Report of Question1-A}
\author{2019xxxxxx 박준영}
\date{}

\begin{document}
\maketitle


\section{Node Type}
기본적인 Binary Search Tree의 node 구조체에 tree에 있는 key의 개수를 나타내는 count 변수를 넣었다.
\\

\begin{center}
	\begin{threeparttable}
		\caption{count가 추가된 node type}
		\begin{tabular}{|c|l|}
			\hline
			이름 & 설명\\
			\hline
			value & node의 값\\
			count & tree에 있는 해당 key의 개수\\
			leftChild & left subtree의 root 노드\\
			rightChild & right subtree의 root 노드\\
		\hline
		\end{tabular}
	\end{threeparttable}
\end{center}

\section{Insertion}
기존 Binary Search Tree의 insert에서 count를 변경하는 부분을 추가했다.
\\

\begin{algorithm}
	\caption{BST with count field insertion}
	\begin{algorithmic}
		\Function{Insert}{root, value}
			\If{root is null}
				\State root becomes a new node with value
				\State root.count $\gets$ 1
			\ElsIf{root.value $>$ value}
				\State root.leftChild = Insert(root.leftChild, value)
			\ElsIf{root.value $<$ value}
				\State root.rightChild = Insert(root.rightChild, value)
			\Else
				\State add 1 to root.count
			\EndIf
			
			\State \Return{root}
		\EndFunction
	\end{algorithmic}
\end{algorithm}
\pagebreak

\section{Deletion}
기존 Binary Search Tree의 delete에서 count를 고려하여 처리하도록 변경했다.
\\

\begin{algorithm}
	\caption{BST with count field deletion}
	\begin{algorithmic}
		\Function{Delete}{root, value}
			\If{root is null}
				\State produce error message
			\ElsIf{root.value $>$ value}
				\State root.leftChild = Delete(root.leftChild, value)
			\ElsIf{root.value $<$ value}
				\State root.rightChild = Delete(root.rightChild, value)
			\ElsIf{root.count $>$ 1}
				\State subtract 1 to root.count
			\ElsIf{root has neither left child nor right child}
				\State set root value to min value of left subtree
				\State root.leftChild = Delete(root.leftChild, root.value)
			\Else
				\If{root don't have left child}
					\State set root to right child node
				\ElsIf{root don't have right child}
					\State set root to left child node
				\EndIf
			\EndIf
		\EndFunction
	\end{algorithmic}
\end{algorithm}

\section{Finding}
기존 Binary Search Tree의 find와 동일하다.

\end{document}
